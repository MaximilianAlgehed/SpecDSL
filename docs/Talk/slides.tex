\documentclass{beamer}

\mode<presentation> {

\usetheme{Madrid}

\setbeamertemplate{footline} % To remove the footer line in all slides uncomment this line

\setbeamertemplate{navigation symbols}{} % To remove the navigation symbols from the bottom of all slides uncomment this line
}

\usepackage{minted} % syntax highlighting for haskell expressions
\usepackage[nounderscore]{syntax}
\usepackage{graphicx} % Allows including images
\usepackage[utf8]{inputenc}

%----------------------------------------------------------------------------------------
%	TITLE PAGE
%----------------------------------------------------------------------------------------

\title[SpecDSL]{SpecDSL \\ \large{Session types + LTL = QuickCheck}}

\author{Maximilian Algehed} % Your name
\institute[CTH] % Your institution as it will appear on the bottom of every slide, may be shorthand to save space
{
Chalmers University of Technology \\ % Your institution for the title page
\medskip
\textit{m.algehed@gmail.com} % Your email address
}
\date{\today} % Date, can be changed to a custom date

\begin{document}

\begin{frame}
    \titlepage % Print the title page as the first slide
\end{frame}

\begin{frame}
    \frametitle{Motivating scenario}
    \begin{columns}
        \column{0.15\textwidth}
        \begin{block}{Server}
            \small{Maybe typed}
        \end{block}

        \column{0.1\textwidth}
        \\
        \centerline{TCP}
        \centerline{\Large{$\implies$}}
        \centerline{\Large{$\impliedby$}}
        \centerline{Definately not typed}

        \column{0.15\textwidth}
        \begin{block}{Client}
            \small{Javascript: Definately not typed}
        \end{block}
        
    \end{columns}
\end{frame}

\begin{frame}
    \frametitle{What we want}
    \centering
    \Large{Property based testing, of course!}
\end{frame}

\begin{frame}
    \frametitle{Communication}
    \Large{A problem in two parts}
    \begin{itemize}
        \item How is the communication happening?
        \item How do the parties behave?
    \end{itemize}
\end{frame}

\begin{frame}[fragile]
    \frametitle{Session Types, but quickly!}
    \large{Simplified GV calculus}
    \\
    \begin{grammar}
        <T> ::= Any serializable type

        <S> ::= !<T> | ?<T> | <S>.<S> | <S> $\wedge$ <S> | <S> $\vee$ <S> | $end$
    \end{grammar}
\end{frame}

\begin{frame}[fragile]
    \frametitle{Example}
    \Large{$client = !ISBN.(client \vee ?Cost.!CreditCardNr).end$}
\end{frame}

\begin{frame}
    \frametitle{Communication}
    \Large{A problem in two parts}
    \begin{itemize}
        \item How is the communication happening? Session Types!
        \item How do the parties behave?
    \end{itemize}
\end{frame}

\begin{frame}[fragile]
    \frametitle{Session Types as a DSL}
    \large{Session types in Haskell}
    \\
    \begin{minted}{haskell}
data ST tu = Send tu
           | Get  tu
           | (ST tu) :. (ST tu)
           | (ST tu) :& (ST tu)
           | (ST tu) :| (ST tu)
           | End
    \end{minted}
    \pause
    But what happened to the \textit{type} bit?
\end{frame}

\begin{frame}
    \frametitle{Summary}
    \begin{itemize}
        \item Session Types provide a natural DSL for communication
        \item LTL lets us have nice properties
        \item Put them togeather and we get QuickCheck for Client-Server communication
    \end{itemize}
\end{frame}

\begin{frame}
    \Huge{\centerline{Questions?}}
\end{frame}

\end{document} 
