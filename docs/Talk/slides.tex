\documentclass{beamer}

\mode<presentation> {

\usetheme{Madrid}

\setbeamertemplate{footline} % To remove the footer line in all slides uncomment this line

\setbeamertemplate{navigation symbols}{} % To remove the navigation symbols from the bottom of all slides uncomment this line
}

\usepackage{minted} % syntax highlighting for haskell expressions
\usepackage[nounderscore]{syntax}
\usepackage{graphicx} % Allows including images
\usepackage[utf8]{inputenc}

%----------------------------------------------------------------------------------------
%	TITLE PAGE
%----------------------------------------------------------------------------------------

\title[SpecDSL]{SpecDSL \\ \large{Session types + LTL = QuickCheck}}

\author{Maximilian Algehed} % Your name
\institute[CTH] % Your institution as it will appear on the bottom of every slide, may be shorthand to save space
{
Chalmers University of Technology \\ % Your institution for the title page
\medskip
\textit{m.algehed@gmail.com} % Your email address
}
\date{\today} % Date, can be changed to a custom date

\begin{document}

\begin{frame}
    \titlepage % Print the title page as the first slide
\end{frame}

\begin{frame}
    \frametitle{Motivating scenario}
    \begin{columns}
        \column{0.15\textwidth}
        \begin{block}{Server}
            \small{Maybe typed}
        \end{block}

        \column{0.1\textwidth}
        \\
        \centerline{TCP}
        \centerline{\Large{$\implies$}}
        \centerline{\Large{$\impliedby$}}
        \centerline{Definately not typed}

        \column{0.15\textwidth}
        \begin{block}{Client}
            \small{Javascript: Definately not typed}
        \end{block}
        
    \end{columns}
\end{frame}

\begin{frame}
    \frametitle{Preliminaries}
    \Large{What this talk is about}
    \begin{itemize}
        \item Testing protocol implementations
        \item Extracting DSLs from logic
    \end{itemize}
    \pause
    \Large{What this talk is \emph{not} about}
    \begin{itemize}
        \item Haskell
        \item Clever type theory (sorry!)
        \item Logic
    \end{itemize}
\end{frame}


\begin{frame}
    \frametitle{What we want}
    \centering
    \Large{Property based testing, of course!}
\end{frame}

\begin{frame}
    \frametitle{A DSL for specifying communication}
    \centering
    \Large{A problem in two parts}
    \begin{itemize}
        \item How is the communication happening?
        \item How do the parties behave?
    \end{itemize}
\end{frame}

\begin{frame}[fragile]
    \frametitle{Session Types, but quickly!}
    \large{Simplified Session Types}
    \\
    \begin{grammar}
        <T> ::= Any serializable type

        <S> ::= !<T> | ?<T> | <S>.<S> | <S> $\wedge$ <S> | <S> $\vee$ <S> | $end$
    \end{grammar}
\end{frame}

\begin{frame}[fragile]
    \frametitle{Example}
    \Large{$client = !ISBN.(client \vee (?Cost.!CreditCardNr)).end$}
    \\~\\
    \large{$server = \overline{client} = ?ISBN.(client \wedge (!Cost.?CreditCardNr)).end$}
\end{frame}

\begin{frame}
    \frametitle{A DSL for specifying communication}
    \centering
    \Large{A problem in two parts}
    \begin{itemize}
        \item How is the communication happening? Session Types!
        \item How do the parties behave?
    \end{itemize}
\end{frame}

\begin{frame}[fragile]
    \frametitle{Linear Temporal Logic, LTL}
    \begin{grammar}
        <$\phi$> ::= p | $\top$ | $\bot$ | $\neg$$\phi$ | $\phi\wedge\phi$ | $X\phi$ | $G\phi$ | $\phi U \phi$ 
    \end{grammar}
\end{frame}

\begin{frame}[fragile]
    \frametitle{LTL semantics}
\end{frame}

\begin{frame}
    \frametitle{A DSL for specifying communication}
    \centering
    \Large{A problem in two parts}
    \begin{itemize}
        \item How is the communication happening? Session Types
        \item How do the parties behave? LTL!
    \end{itemize}
\end{frame}

\begin{frame}[fragile]
    \frametitle{Session Types as a DSL}
    \large{Session types in Haskell}
    \\
    \begin{minted}{haskell}
data ST tu = Send tu
           | Get  tu
           | (ST tu) :. (ST tu)
           | (ST tu) :& (ST tu)
           | (ST tu) :| (ST tu)
           | End
    \end{minted}
    \pause
    But what happened to the \textit{type} bit?
\end{frame}

\begin{frame}[fragile]
    \frametitle{LTL as a DSL}
    \centering
    Shallow
    \begin{minted}{haskell}
type LTL a = [a] -> Bool
    \end{minted}
    \pause
    \centering
    Deep
    \begin{minted}{haskell}
data LTL a = Atomic (a -> LTL a)
           | Top 
           | Bottom
           | (LTL a) And (LTL a)
           | Not (LTL a)
           | X (LTL a)
           | G (LTL a)
           | (LTL a) U (LTL a)
           | Terminated

check :: LTL a -> [a] -> Bool
check = ...
    \end{minted}
\end{frame}

\begin{frame}
    \frametitle{Demo!}
    \Large{That's quite enough background, demo time!}
\end{frame}

\begin{frame}[fragile]
    \frametitle{OK, so how does it work?}
    \centering
    \Large{Communication}
    \small{
    \begin{minted}{haskell}
data Choice = L | R

data Trace tu ut = Terminate
                 | Send ut
                 | Get tu
                 | Br (Trace tu ut) (Trace tu ut)
                 | Ch Choice (Trace tu ut)

type Implementation tu = SessionType tu -> Gen (Trace tu ut)

type Checker = tu -> ut -> Bool
    \end{minted}
}
\end{frame}

\begin{frame}[fragile]
    \frametitle{OK, so how does it work?}
    \centering
    \Large{Communication}
    \begin{minted}{haskell}
data Protocol a = Pure a
                | ChooseLeft
                | ChooseRight
    \end{minted}
    \begin{minted}{haskell}
class BiCh ch a where
    newCh :: IO (ch a)
    putCh :: ch a -> Protocol a -> IO ()
    getCh :: ch a -> IO (Protocol a)
    \end{minted}
\end{frame}

\begin{frame}
    \frametitle{Summary}
    \begin{itemize}
        \item Session Types provide a natural DSL for communication
        \item LTL lets us have nice properties
        \item Put them togeather and we get QuickCheck for Client-Server communication
    \end{itemize}
\end{frame}

\begin{frame}
    \frametitle{Future work}
    \begin{itemize}
        \item Shrinking
        \item Distributed systems
        \item Paper...
    \end{itemize}
\end{frame}
\begin{frame}
    \Huge{\centerline{Questions?}}
\end{frame}

\end{document} 
